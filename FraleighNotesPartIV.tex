\documentclass[12pt, letterpaper]{article}
\usepackage{amssymb}
\usepackage{amsmath}
\usepackage{mathrsfs}

\begin{document}


\title{A First Course in Abstract Algebra by Fraleigh 7th Edition (Notes Part IV)}
\author{Hubert Farnsworth }  
\begin{titlepage}
\maketitle
\end{titlepage}
 

 
\section{Part IV : Rings and Fields}


\subsection{Section 18 : Rings and Fields}


\noindent \underline{\bf 18.1 Definition} A {\bf ring} $\langle R, +, \cdot \rangle$ is a set $R$ together with two binary operations $+$ and $\cdot$, which we call addition and multiplication, defined on $R$ such that the following axioms are satisfied: \\

$\mathscr{R}_1$. $\langle R, + \rangle$ is an abelian group. \\

$\mathscr{R}_2$. Multiplication is associative. \\

$\mathscr{R}_3$. For all $a,b,c \in R$, the {\bf left distributive law}, $a \cdot (b + c) = (a\cdot b) + (a\cdot c)$ and the {\bf right distributive law} $(a+b) \cdot c = (a\cdot c) + (b \cdot c)$ hold. \\

\noindent \underline{\bf 18.8 Theorem} If $R$ is a ring with additive identity $0$, then for any $a,b \in R$ we have \\

1. $0a = a0 = 0$. \\

2. $a(-b) = (-a)b = -(ab)$. \\

3. $(-a)(-b) = ab$. \\

\noindent \underline{\bf 18.9 Definition} For rings $R$ and $R'$, a map $\phi : R \rightarrow R'$ is a {\bf homomorphism} if the following two conditions are satisfied for all $a,b \in R$: \\

1. $\phi(a+b)= \phi(a) + \phi(b)$. \\

2. $\phi(ab) = \phi(a) \phi(b)$. \\

\noindent \underline{\bf 18.12 Definition} An {\bf isomorphism} $\phi : R \rightarrow R'$ from a ring $R$ to a ring $R'$ is a homomorphism that is one to one and onto $R'$. The rings $R$ and $R'$ are then {\bf isomorphic}. \\

\noindent \underline{\bf 18.14 Definition} A ring in which multiplication is commutative is a {\bf commutative ring}. A ring with a multiplicative identity element is a {\bf ring with unity}; the multiplicative identity element 1 is called "{\bf unity}". \\

\noindent \underline{\bf 18.16 Definition} Let $R$ be a ring with unity $1 \neq 0$. An element $u \in R$ is a {\bf unit} of $R$ if it has a multiplicative inverse in $R$. If every nonzero element of $R$ is a unit, then $R$ is a {\bf division ring} (or {\bf skew field}). A {\bf field} is a commutative division ring. A noncommutative division ring is called a "{\bf strictly skew field}". \\

\noindent \underline{\bf Definition} If we have a set, together with certain specified type of algebraic structure, then any subset of this set, together with a natural induced algebraic structure that yields an algebraic structure of the same type is a substructure. (group - subgroup, ring - subring, field - subfield, etc.) \\

\noindent \underline{\bf Notable Exercises} \\

5) Compute $(2,3)(3,5)$ in $\mathbb{Z}_5 \times \mathbb{Z}_9$. \\

(Answer) : We use the familiar properties of $\mathbb{Z}_n$ along with the definitions given for multiplication in this section to get $2 \cdot 3 = 1$ in $\mathbb{Z}_5$ and $3 \cdot 5 = 6$ in $\mathbb{Z}_9$, so $(2,3)(3,5) = (1,6) \in  \mathbb{Z}_5 \times \mathbb{Z}_9$. \\

15) Describe all units in $\mathbb{Z} \times \mathbb{Z}$. \\

(Answer) : We know that $1,-1$ are the only units in $\mathbb{Z}$ with $1 \cdot 1 = 1, (-1) \cdot (-1) = 1$. From this we see that the units in $\mathbb{Z}\times \mathbb{Z}$ are $(1,1),(-1,-1),(1,-1),(-1,1)$ (note that each unit is its own multiplicative inverse as well). \\

17) Describe all the units in $\mathbb{Q}$. \\

(Answer) : The units in $\mathbb{Q}$ are all the elements of $\mathbb{Q}^*$ (all nonzero rational numbers). \\

19) Describe all the units in $\mathbb{Z}_4$. \\

(Answer) : The units in $\mathbb{Z}_4$ are 1 and 3 with (1)(1) = 1 and (3)(3) = 1 in this ring (note that 1 and 3 as integers are relatively prime to the integer 4, which is another way to know which elements of $\mathbb{Z}_4$ are units). \\

31) Give an example of a ring having two elements $a,b$ such that $ab = 0$ but neither $a$ nor $b$ is zero. \\

(Answer) : One example is the ring $M_2(\mathbb{R})$ where the zero element is the 2 by 2 matrix with all entries $0 \in \mathbb{R}$. A possible choice for $a,b$ is 
$$ a = \begin{pmatrix}
1 & -1 \\ 1 &-1
\end{pmatrix}\; , \; b = \begin{pmatrix}
1 &1  \\ 1 &1
\end{pmatrix}\; \mbox{ so that } ab = \begin{pmatrix}
1 & -1 \\ 1 &-1
\end{pmatrix}\begin{pmatrix}
1 &1  \\ 1 &1
\end{pmatrix} = \begin{pmatrix}
0 & 0 \\ 0 &0
\end{pmatrix} \;.$$

33) Mark the following statements as true or false. \\

(Answers) : \\

a. Every field is also a ring. -- True. The definition of a field given requires the set in question to be a ring with the few added requirements of every nonzero element having a multiplicative inverse and also commutative multiplication. \\

c. Every ring has at least two units. -- False. Consider the ring $\mathbb{Z}_2$ with unity 1 and also 1 as the only unit. \\

e. It is possible for a subset of a field to be a ring but not a subfield, under the induced operations. -- True. Consider the field $\mathbb{Q}$ with subset $\mathbb{Z} \subset \mathbb{Q}$. Then $\mathbb{Z}$ is a ring but not a field under the usual operations applied to both $\mathbb{Q},\mathbb{Z}$. \\

g. Multiplication in a field is commutative. -- True. This is required by the definition. \\

i. Addition in a ring is commutative. -- True. This is required because of the first ring axiom. \\

\subsection{Section 19 : Integral Domains}

\noindent \underline{\bf 19.2 Definition} If $a,b$ are two nonzero elements of a ring $R$ such that $ab = 0$, then $a,b$ are {\bf divisors of 0} (or {\bf 0 divisors}). \\

\noindent \underline{\bf 19.3 Theorem} In the ring $\mathbb{Z}_n$, the divisors of $0$ are precisely those nonzero elements that are not relatively prime to $n$. \\

\noindent \underline{\bf 19.4 Corollary} If $p$ is prime, then $Z_p$ has no divisors of $0$. \\

\noindent \underline{\bf 19.5 Theorem} The cancellation laws hold in a ring $R$ if and only if $R$ has no divisors of $0$. \\

\noindent \underline{\bf 19.6 Definition} An {\bf integral domain} $D$ is a commutative ring with unity $1 \neq 0$, and containing no divisors of $0$. \\

\noindent \underline{\bf 19.9 Theorem} Every field $F$ is an integral domain. \\

\noindent \underline{\bf 19.11 Theorem} Every finite integral domain is a field. \\

\noindent \underline{\bf 19.12 Corollary} If $p$ is prime, then $\mathbb{Z}_p$ is a field. \\

\noindent \underline{\bf 19.13 Definition} If for a ring $R$, a positive integer $n$ exists such that $n \cdot a = 0$ for all $a \in R$, then the least such positive integer is the {\bf characteristic of the ring $R$}. If no such positive integer exists, then $R$ is of {\bf characteristic $0$}. \\

\noindent \underline{\bf 19.15 Theorem} Let $R$ be a ring with unity. If $n \cdot 1 \neq 0 \; \forall n \in \mathbb{N}$, then $R$ has characteristic 0. If $n \cdot 1 = 0 $ for some $n \in \mathbb{N}$, then the smallest such $n$ is  the characteristic of $R$. \\

\noindent \underline{\bf Notable Exercises}\\

3) Find all solutions of the equation $x^2 + 2x +2 = 0$ in $\mathbb{Z}_6$. \\

(Answer) : We can see that there are no solutions by plugging in the 6 elements of $\mathbb{Z}_6$ in for $x$ in $x^2 + 2x + 2$ and finding that the result is never 0. That is, $0^2 + 2(0) + 2 = 2 \neq 0, 1^2 +2(1) + 2 = 5 \neq 0, ... 5^2 + 2(5) + 2 = 25 + 10 + 2 = 1 + 4 +2 = 1 \neq 0$. \\

7) Find the characteristic of the ring $R = \mathbb{Z}_3 \times 3 \mathbb{Z}$. \\

(Answer) : This ring is of characteristic 0. To see why, suppose that for $(a,b) \in R$, that $n \cdot (a,b) = (0,0)$, since the zero element of the ring is $(0,0)$. But then this would require $n \cdot b = 0$ in $3\mathbb{Z}$. Since $3\mathbb{Z} \subset \mathbb{Z}$ this can only occur if $b = 0$, but we require some $n$ such that $n \cdot b = 0$ for any $b \in 3 \mathbb{Z}$. So we conclude that $R$ must be of characteristic 0 because there can be no $n \in \mathbb{N}$ such that $n \cdot (a,b) = (0,0)$ for all $(a,b) \in R$. \\

Find the characteristic of the ring $R = \mathbb{Z}_3 \times \mathbb{Z}_4$. \\

(Answer) : Note that $R$ has unity $(1,1)$. Then by Theorem 19.15 if we can find the smallest $n \mathbb{N}$ such that $n \cdot (1,1) = (0,0)$, then $R$ must be of characteristic $n$ (If no finite $n$ satisfies this then we conclude $R$ is of characteristic 0). We can compute by hand to check our work, but some thought shows that $n = lcm(3,4) = 12$ since in this case $n \cdot (1,1) = (12,12) = (0,0)$ and $12$ is the least positive integer such that we have a multiple of both 3 and 4. Thus, $R$ is of characteristic 12. \\

13) Let $R$ be a commutative ring with unity and of characteristic 3. Let $a,b \in R$. Compute and simplify $(a+b)^6$. \\

(Answer) : $$(a+b)^6 = a^6 + 6a^5b + 15a^4b^2 + 20a^3b^3 + 15a^2b^4 + 6ab^5 + b^6 $$ $$= a^6 + 0 + 0 + ((6)(3) + 2)a^3b^3 + 0 + 0 + b^6 = a^6 + 2a^3b^3 + b^6\;.$$

Here we have used the fact that if $a,b \in R$, then $a^pb^q \in R$ for any $p,q \in \mathbb{N} \cup \{0\}$ and $3r = 0$ for any $r \in R$, so that $3kr = k \cdot 0 = 0$ for any integer $k$. \\

17) Mark the following statements as true or false. \\

(Answers) : \\

g. The direct product of two integral domains is also an integral domain. -- False. As one counterexample, consider the direct product of integral domains $\mathbb{Z} \times \mathbb{Z}$. Here $(1,0)(0,1) = (0,0)$ while $(1,0),(0,1)$ are nonzero elements. Therefore $\mathbb{Z} \times \mathbb{Z}$ contains zero divisors. \\

i. $n\mathbb{Z}$ is a subdomain of $\mathbb{Z}$. -- False. We assume here that subdomain refers to a sub- integral domain, which is not easy to write in a sensible way. By definition, an integral domain must contain a unity element. But $n\mathbb{Z}$ contains a unity element only if it contains $1 \in \mathbb{Z}$, which is only true if $n = 1$. For any other $n$, we fail to meet the conditions defining an integral domain. \\

\subsection{Section 20 : Fermat's and Euler's Theorems}

\noindent \underline{\bf 20.1 Theorem} {\bf (Fermat's Little Theorem} If $a \in \mathbb{Z}$ and $p$ is a prime not dividing $a$, the $p$ divides $a^{p-1} - 1$, that is, $a^{p-1} \equiv 1 \,(\mbox{mod } p)$ for $a \not\equiv 0 \, (\mbox{mod } p)$. \\

\noindent \underline{\bf 20.2 Corollary} If $a \in \mathbb{Z}$, then $a^p \equiv a \, (\mbox{mod } p)$ for any prime $p$. \\

\noindent \underline{\bf 20.6 Theorem} The set $G_n$ of nonzero elements of $\mathbb{Z}_n$ that are not 0 divisors form a group under multiplication modulo $n$. \\

\noindent \underline{\bf Definition}  The function $\phi : \mathbb{N} \rightarrow \mathbb{N}$, where $\phi(n)$ is the number of positive integers less than or equal to $n$, is called the {\bf Euler - phi function}. \\

\noindent \underline{\bf 20.8 Theorem} {\bf(Euler's Theorem)} If $a$ is an integer relatively prime to $n$, then $n$ divides $a^{\phi(n)} - 1$, that is, $a^{\phi(n)} \equiv 1 \, (\mbox{mod } n)$. \\

\noindent \underline{\bf 20.10 Theorem} Let $m$ be a positive integer and let $a \in \mathbb{Z}_m$ be relatively prime to $m$. For each $b \in \mathbb{Z}_m$, the equation $ax = b$ has a unique solution in $\mathbb{Z}_m$. \\

\noindent \underline{\bf 20.11 Corollary} If $a$ and $m$ are relatively prime integers, then for any integer $b$, the congruence $ax \equiv b \, (\mbox{mod } m)$ has as solutions all integers in precisely one congruence class modulo $m$. \\

\noindent \underline{\bf 20.12 Theorem} Let $m$ be a positive integer and let $a,b \in \mathbb{Z}_m$. Let $d = gcd(a,m)$. The equation $ax = b$ has a solution in $\mathbb{Z}_m$ if and only if $d$ divides $b$. When $d$ does divide $b$, the equation has exactly $d$ solutions in $\mathbb{Z}_m$.  \\

\noindent \underline{\bf 20.13 Corollary} Let $d = gcd(a,m)$. The congruence $ax \equiv b \, (\mbox{mod } m)$ has a solution if and only if $d$ divides $b$. When this is the case, the solutions are the integers in exactly $d$ distinct residue classes modulo $m$. \\

\noindent \underline{\bf Notable Exercises} \\

1) Find a generator for the multiplicative group of nonzero elements of the field $\mathbb{Z}_7$. \\

(Answer) : $$\langle 1 \rangle = \{1\} $$
(1 is not a generator)$$ \langle 2 \rangle = \{2,4,1\}$$
(2 is not a generator) $$ \langle 3 \rangle = \{3,2,6,4,5,1\} = \mathbb{Z}_7 - \{0\}$$
(3 is a generator)$$ \langle 4 \rangle = \{4,2,1\}$$
(4 is not a generator) $$ \langle 5 \rangle = \{5,4,6,2,3,1\} = \mathbb{Z}_7 - \{0\}$$
(5 is a generator)$$\langle 6 \rangle = \{6,1\}$$
(6 is not a generator) \\

5) Use Fermat's Theorem to find the remainder of $37^{49}$ when divided by 7. \\

(Answer) : Here we take $a = 37$ and $p = 7$ as described in Fermat's Theorem, which applies since 7 does not divide 37. We know that $37^6 \equiv 1 \, (\mbox{mod } 7)$. From this we have $$37^{49} = (37^6)^837 \equiv (1)(37) \equiv 37 \equiv 2 \, (\mbox{mod } 7)\;.$$ 

9) Compute $\phi(pq)$ where $p$ and $q$ are both primes (and $\phi$ is the Euler-phi function. \\

(Answer) : There are $pq - 1$ positive integers less than $pq$. Since $p$ is prime, the only positive integers $k$ that are less than $pq$ such that $gcd(pq,k) = p$ are the the multiples of $p$, and there are $q-1$ of these. Similarly there are $p-1$ multiples of $q$ so that $gcd(pq,k) = q$. All other positive integers less than $pq$ are relatively prime to $pq$. So we are left with $pq - 1 - (p-1) - (q-1) = (p-1)(q-1)$. \\

23) Mark the following statements as true or false. \\

a. $a^{p-1} \equiv 1 \, (\mbox{mod } p)$ for all integers $a$ and primes $p$. -- False. This is the result of Fermat's Theorem, without including the condition that $p$ does not divide $a$. We see that indeed this condition is necessary. As a counterexample to this statement, consider prime $p = 3$ and $a = 6$, so that $p | a$. Then we should have $6^{3-1} \equiv 1 \, (\mbox{mod } 3)$, but this is not true since $36 \equiv 0 \, (\mbox{mod } 3)$. \\

g. The product of two nonunits in $\mathbb{Z}_n$ may be a unit. -- False. The units in $\mathbb{Z}_n$ are precisely the positive integers less than $n$ that are relatively prime to $n$. So if we have two nonunits, say $a,b$, then $a,b$ are not relatively prime to $n$. Let $gcd(a,n) = d_1 > 1$ and $gcd(b,n) = d_2 >1$. Consider the product $ab = kd_1d_2$ for some $k \in \mathbb{Z}$. Then $gcd(ab,n) = max \{d_1,d_2\} > 1$, so that $ab$ is not a unit in $\mathbb{Z}_n$. (Not really confident at all in this proof). \\

i. Every congruence $ax \equiv b \, (\mbox{mod } p)$, where $p$ is prime, has a solution. -- False. This is Corollary 20.11 without the requirement that $a,m$ (where $p$ takes the place of $m$ in the corollary) be relatively prime. For a counterexample consider $2x \equiv 1 \, (\mbox{mod } 2)$, which is solvable if and only if $2 | 2x - 1$ (where $x$ is an integer). But $2x - 1$ is odd for any integer $x$, so that $2x - 1$ can never be divisible by $2$. So we conclude that the congruence equation has no integer solutions. \\

\subsection{Section 21 : The Field of Quotients of an Integral Domain}

Let $D$ be an integral domain. We refer to $D$ and the subset of $D \times D$ given by $S = \{(a,b) \; | \; a,b \in D, b \neq 0\}$ in what follows as given here unless otherwise specified. \\

\noindent \underline{\bf 21.1 Definition} Two elements $(a,b), (c,d) \in S$ are {\bf equivalent}, denoted by $(a,b) \sim (c,d)$, if and only if $ad = bc$. \\

\noindent \underline{\bf 21.2 Lemma} The relation $\sim$ from the above definition is an equivalence relation on $S$. \\

Note : To prove this lemma, it is very important that the integral domain $D$ is commutative (which is required by definition of integral domain). \\

\noindent \underline{\bf Definition} $$[(a,b)] = \{(c,d) \in S \; | \; (a,b) \sim (c,d)\}\,.$$

\noindent \underline{\bf 21.3 Lemma} Let $F$ be the set of all equivalence classes $[(a,b)]$ for $(a,b) in S$. For $[(a,b)], [(c,d)]$ in $F$, the equations $$[(a,b)] + [(c,d)] = [(ad+bc,bd)]$$ and $$[(a,b)][(c,d)]$$ give well defined operations of addition and multiplication on $F$. \\

\noindent \underline{\bf 21.4 Lemma} The map $i : D \rightarrow F$ given by $i(a) = [(a,1)]$ is an isomorphism of $D$ with a subring of $F$. \\

\noindent \underline{\bf 21.5 Theorem} Any integral domain $D$ can be enlarged to (or embedded in) a field $F$ such that every element of $F$ can be expressed as the quotient of two elements of $D$. (Such a field $F$ is a {\bf field of quotients of} $D$). \\

\noindent \underline{\bf 21.6 Theorem} Let $F$ be a field of quotients of $D$ and let $L$ be any field containing $D$. Then there exists a map $\psi : F \rightarrow L$ that gives an isomorphism of $F$ with a subfield of $L$ such that $\psi(a) = a$ for $a \in D$. \\

\noindent \underline{\bf 21.8 Corollary} Every field $L$ containing an integral domain $D$ contains a field of quotients of $D$.  \\

\noindent \underline{\bf 21.9 Corollary} Any two fields of quotients of an integral domain $D$ are isomorphic. \\

\noindent \underline{\bf Notable Exercises}
 
\end{document}